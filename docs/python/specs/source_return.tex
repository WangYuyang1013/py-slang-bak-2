\subsection*{Restrictions}

\begin{itemize}
\item In Python, a logical line is the smallest unit of code that the 
interpreter can execute, typically corresponding to a complete statement. 
A physical line, by contrast, is a single line in the source file, 
usually ending with a newline character. A logical line may span multiple 
physical lines depending on whether line joining is used.

By default, each physical line corresponds to a logical line. However, 
when a statement is too long, Python allows it to be split across multiple 
physical lines using line continuation. Python provides two ways to join 
lines: \textbf{explicit line joining} and \textbf{implicit line joining}.

\textbf{Explicit line joining} uses a backslash (\textbackslash) to join 
the current physical line with the next one, forming a single logical line. 
The backslash must be the last character on the line—no characters 
(including whitespace or comments) are allowed after it.

\begin{lstlisting}
total = 1 + 2 + 3 + \
        4 + 5
\end{lstlisting}

\textbf{Implicit line joining} occurs when expressions are enclosed in 
parentheses \texttt{()}, square brackets \texttt{[]}, or braces \texttt{\{\}}. 
Within these delimiters, line breaks are allowed without using a backslash.

\begin{lstlisting}
total = (
            1 + 2 + 3 +
            4 + 5
         )
\end{lstlisting}

Therefore, when breaking a statement across lines, you must either use a backslash for explicit continuation or wrap the expression in parentheses, brackets, or braces for implicit continuation. Unlike JavaScript, which uses automatic semicolon insertion, Python requires each line to be syntactically complete—it does not infer the end of a statement implicitly.

For details on logical lines, see the official Python documentation:
\href{https://docs.python.org/3/reference/lexical_analysis.html#logical-lines}{Python Lexical Analysis – Logical Lines}.


\item Return statements are only allowed in bodies of functions. Each return statement must appear on a single logical line.
\item Lambda expressions are limited to a single logical line.
\item Re-declaration variables or functions is not allowed. Once a variable or function is defined, it cannot be redefined with the same name in the same scope
\footnote{
Scope refers to the region of a program in which a particular name (such as a variable, function, or class) is defined and can be accessed. In other words, 
it determines the part of the program where you can use that name without causing a name error. 
Scope is determined by the program's structure (usually its lexical or textual layout) and governs the visibility and lifetime of variables and other identifiers.

In \href{https://docs.python.org/3/tutorial/classes.html\#python-scopes-and-namespaces}{\color{DarkBlue}Python}, the \emph{scope} of a declaration is determined lexically: a variable declared inside a function is local to that function; 
if it is declared outside any function, it is global (i.e., module-level). Moreover, if a variable is declared in an enclosing function, 
it is available to inner functions (the enclosing scope), and if not found in these scopes, Python looks into the built-in scope.
}.
\end{itemize}


